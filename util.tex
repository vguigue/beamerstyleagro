

\newcommand{\transp}[2][35]{\color{fg!#1}#2}


\newcommand\warning{\includegraphics[width=.7cm,keepaspectratio]{./logos/warning}}
\def\plus{\color{green!80!black} $\mathbf +$}
\def\moins{\color{red} $\mathbf -$}
\def\mix{\color{orange} $\mathbf \approx$}
\newcommand\bibref[4][1]{
\hspace*{-0.5cm}   
\begin{tabular}{m{0.6cm}m{#1\linewidth}}
    \includegraphics[width=.7cm,keepaspectratio]{./logos/bib}  & 
      \scriptsize  {\color{black} \textit{#2}}, #3 \newline #4  
   \end{tabular}
   }
\newcommand\bibrefR[4][1]{
\vspace*{-4mm}
\hspace*{-0.5cm}   
\begin{tabular}{m{0.2cm}m{#1\linewidth}}
   \vspace*{2mm} \includegraphics[width=.4cm,keepaspectratio]{./logos/bib}  & 
      \tiny  {\color{black} \textit{#2}},  #4  , #3
    \end{tabular}
    }
    

\def\espace{\hspace*{5mm}}

\def\cov{\mbox{cov}}




% \def\Y{\mathcal{Y}}
% \def\E{\mathbb{E}}
% \def\R{\mathbb{R}}
% \def\N{\mathbb{N}}
% \def\X{\mathcal{X}}
\def\cX{\mathcal{X}}
\def\F{\mathcal{F}}
% \def\cR{\mathcal{R}}
% \def\bx{\mathbf{x}}
% \def\by{\mathbf{y}}
% \def\bw{\mathbf{w}}
% \def\cN{\mathcal{N}}
% 

\def\R{\mathbb{R}}
\def\C{\mathbb{C}}
\def\N{\mathbb{N}}
\def\Q{\mathbb{Q}}
\def\Z{\mathbb{Z}}

\def\esp{\mathbb{E}}
\def\var{\mathbb{V}}
\def\P{\mathbb{P}}


\def\mA{\mathcal{A}}
\def\mB{\mathcal{B}}
\def\mP{\mathcal{P}}
\def\mN{\mathcal{N}}
\def\mF{\mathcal{F}}
\def\mH{\mathcal{H}}


\def\bz{\mathbf{z}}
\def\bZ{\mathbf{Z}}

\def\bi{\mathbf{i}}
\def\bu{\mathbf{u}}
\def\bx{\mathbf{x}}
\def\bX{\mathbf{X}}
\def\by{\mathbf{y}}
\def\bY{\mathbf{Y}}
\def\bz{\bm{z}}
\def\bZ{\bm{Z}}
\def\bbeta{\mathbf{\beta}}
\renewcommand{\epsilon}{\varepsilon}


\def\TBE{\emph{\textsc{To be extended.}}}
\def\TBD{\emph{\textsc{To be done.}}}
\def\EOP{\emph{\textsc{End Of Proof}}}

%%%%% les petits carrés en fin de démo.
% \qed is a small square used at the end of proofs
% \fiqed is a small FILLED square used at the end of proofs
% \fiqedsym is the square part from \fiqed


% \def\qed% for open box with right adjust:
% {\strut\hfill\lower0.5\baselineskip\vbox{\hrule width
% 0.5em\nointerlineskip
%                         \hbox to 0.5em{\vrule height 0.5em
%                              \hfill
%                              \vrule height 0.5em}
%                     \nointerlineskip
%                     \hrule width 0.5em}\bigbreak}
% \def\qedsym% for open box symbol only:
% {\vbox{\hrule width 0.5em\nointerlineskip
%                         \hbox to 0.5em{\vrule height 0.5em
%                              \hfill
%                              \vrule height 0.5em}
%                     \nointerlineskip
%                     \hrule width 0.5em}}
% \def\fiqed% for filled box with right adjust:
% {\strut\hfill\lower0.5\baselineskip\hbox{\vrule height 1.5ex
% width 1.5ex
%                                                depth -.1ex}\bigbreak}

% \def\fiqedsym% for filled box symbol only:
% {\vrule height 1.5ex width 1.5ex depth -.1ex}

% et ceux de Vincent

\newcommand\vfillb{\vskip0ptplus1filll\relax}

\newcommand{\myarr}[1]{
    \begin{array}{ccccccccc} #1
    \end{array}
}
\newcommand{\matrice}[1]{
   \left[ \begin{array}{ccccccccc} #1
    \end{array}
    \right]
}
\newcommand{\TOSEE}[1]{ {\red
\begin{pspicture}(0,0)
  \rput(-2.3,0){
    \begin{minipage}[h]{2.5cm}
      #1
    \end{minipage}
}
\end{pspicture}}
}

\newcommand{\im}{\mbox{Im}}


\newcommand{\st}{\mbox{s.t. }}
\newcommand{\tq}{\mbox{tel que }}

%%\usepackage{float}
%%\floatstyle{ruled}
%%\newfloat{algo}{htbp}{alg}[section]
%%\floatname{algo}{Alg.} % titre du caption
\newcommand{\sign}{\mbox{sign}}
\newcommand{\energy}{\mathcal{E}}
\newcommand{\diag}{\mbox{diag}}

\newcommand\bw{\mathbf{w}}
%\newcommand\bx{\mathbf{x}}

\newcommand{\bbun}{\mbox{\textbb 1}}
\newcommand{\bbN}{\mathbb{N}}
\newcommand{\bbC}{\mathbb{C}}
\newcommand{\bbR}{\mathbb{R}}
\newcommand{\bbQ}{\mathbb{Q}}
\newcommand{\bbZ}{\mathbb{Z}}
\newcommand{\bbP}{\mathbb{P}}
\newcommand{\bbK}{\mathbb{K}}

\newcommand{\eme}{^{\grave{e}me}}

\newcommand{\calA}{\mathcal{A}}
\newcommand{\calB}{\mathcal{B}}
\newcommand{\calC}{\mathcal{C}}
\newcommand{\calD}{\mathcal{D}}
\newcommand{\calE}{\mathcal{E}}
\newcommand{\calF}{\mathcal{F}}
\newcommand{\calG}{\mathcal{G}}
\newcommand{\calH}{\mathcal{H}}
\newcommand{\calI}{\mathcal{I}}
\newcommand{\calK}{\mathcal{K}}
\newcommand{\calL}{\mathcal{L}}
\newcommand{\calM}{\mathcal{M}}
\newcommand{\calN}{\mathcal{N}}
\newcommand{\calR}{\mathcal{R}}
\newcommand{\calS}{\mathcal{S}}
\newcommand{\calU}{\mathcal{U}}
\newcommand{\calW}{\mathcal{W}}

\newcommand{\calT}{\mathcal{T}}
\newcommand{\calX}{\mathcal{X}}
\newcommand{\calY}{\mathcal{Y}}

 \DeclareMathOperator*{\argmin}{arg\,min}
\DeclareMathOperator*{\argmax}{arg\,max}

\newcommand{\intall}{\int_{-\infty}^{+\infty}}

\newcommand{\e}{\mbox{e}}
\newcommand{\dist}{\mbox{dist}}
\newenvironment{remarquetmp}{\\ \color{red}}{\\}
\newcommand{\ra}{\rightarrow}
\newcommand{\la}{\leftarrow}
\newcommand{\ssi}{\Longleftrightarrow}
\newcommand{\implique}{\Longrightarrow}
\newcommand{\Ra}{\Longrightarrow}
\newcommand{\card}{\mbox{card}}
\newcommand{\asin}{\mbox{asin}}
\newcommand{\acos}{\mbox{acos}}
\newcommand{\second}{{\prime\prime}}
\newcommand{\tierce}{{\prime\prime\prime}}
\newcommand{\tsp}{^{\small T}}

\newcommand{\bul}{$\bullet$ }

\newcommand{\tocheck}[1]{{\red \emph{#1}} }

\newcommand{\vect}[1]{\overrightarrow{#1}}
\newcommand{\1}{\mbox{\textbb 1}}


\newcommand{\err}{\calC} %{\mbox{err}}

\newcommand{\egint}{\overset{\scriptsize ?}{=}}

%\newcommand\figwidth[2]{\includegraphics[width=#1\textwidth]{#2}}
%\newcommand\figheight[2]{\includegraphics[height=#1\textwidth]{#2}}
%
%\newcounter{Exo}[section]
%
%\newenvironment{disarray}%
% {\everymath{\displaystyle\everymath{}}\array}%
% {\endarray}
%
%%\makeatletter
%%\newcommand{\exercice}[1]{\stepcounter{Exo} {-3.5ex \@plus -1ex \@minus -.2ex} \textbf{\large Exercice \theExo~: #1} {2.3ex \@plus.2ex}}
%%\makeatother
%\newcommand{\espacereponse}[1]{
%\noindent
%\fbox{
%  \begin{minipage}{1.0\linewidth}
%    \vspace*{#1}
%
%
%    \hspace*{1.0\linewidth}
%  \end{minipage}
%
%}}

%\newcommand{\lignehorcentre}{
%\noindent
%%\linethickness{2}
%\begin{picture}(0,0)(0,0)
% %\psline(0mm,3mm)(15.5cm,3mm) %
%  \line(1,0){450}
%\end{picture}\\
%}
%\newcommand{\myhline}{
%%\vspace*{-0.5cm}
%\noindent
%\begin{picture}(0,-12)(0,-12)
%  \line(1,0){450}
%\end{picture} 
%%\vspace*{-3mm}
%}

%\makeatletter
% \renewcommand\subsection{\@startsection{subsection}{2}{\z@}%
% 	{-3.5ex \@plus -1ex \@minus -.2ex}%
% 	{2.3ex \@plus.2ex}%
% 	{\reset@font\Large\bfseries}}
%\makeatother
%
%\newcommand{\exercice}[1]{\stepcounter{Exo} \subsection*{\normalsize Exercice \theExo~: #1\\ 
%~
%\vspace*{-0.8cm}\\ 
%%
%\lignehorcentre} 
%\vspace*{-5mm}}
%
%\newcommand{\sansexercice}[1]{\subsection*{\normalsize  #1\\ 
%~
%\vspace*{-0.8cm}\\ 
%%
%\lignehorcentre} 
%\vspace*{-5mm}}


% \newcommand{\correction}[1]{
% \noindent \textbf{Correction~: }  
% %\noindent 
% \begin{center}
%   \begin{minipage}{0.75\textwidth}
%     #1
%   \end{minipage}
% \end{center}
% } 
%\newcommand{\annexe}[1]{ \subsection*{\normalsize Annexe~: #1\\ 
%~
%\vspace*{-0.8cm}\\ 
%%
%\lignehorcentre} 
%\vspace*{-5mm}}
%
% 
%\newcommand{\TF}{\mathcal{T_F}}
%\newcommand{\ITF}{\mathcal{T_F}^{-1}}
%
%\newenvironment{remark} {
%   \begin{center}
%     \begin{minipage}{15cm}
%     }{
%     \end{minipage}
%   \end{center}
%}
%
%
%
%\newsavebox{\fmbox}
%\newenvironment{exemple}
%     {
%\noindent\begin{lrbox}{\fmbox}\begin{minipage}{\textwidth}}
%     {\end{minipage}\end{lrbox}\fbox{\usebox{\fmbox}}}
%\newenvironment{rappel}
%     {\vspace*{10pt} \noindent\begin{lrbox}{\fmbox}\begin{minipage}{\textwidth}}
%     {\end{minipage}\end{lrbox}\fbox{\usebox{\fmbox}} \vspace*{10pt}}
%
%\newcommand{\func} [1]{f(#1)}
%
%
